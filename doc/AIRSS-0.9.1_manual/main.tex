%! TEX program = xelatex

% Author: Yang Li
% Date: 2020.10.24-2020.10.28
% copyleft liyang@Tsinghua Univ.
%

\documentclass[a4paper, 10pt]{article}
%===================== Common Set ======================
\usepackage{xeCJK}
\usepackage[colorlinks=true]{hyperref}
\usepackage{amsmath}
\usepackage{indentfirst}
\setlength{\parindent}{2em}
%===================== Table Env. ======================
\usepackage{float}
\usepackage{booktabs}
\usepackage{longtable}
\usepackage{array}
%===================== List Env. =======================
\usepackage[T1]{fontenc}
\usepackage[charter]{mathdesign}
\usepackage{eulervm}
\usepackage{mathtools,enumitem,lipsum}
\newlist{maineu}{enumerate}{1}
\setlist[maineu]{label=\arabic*.,align=right,leftmargin=2em,itemsep=2pt}
\newlist{subeu}{enumerate}{2}
\setlist[subeu]{label=(\alph*),align=right,widest=i,labelsep=3pt,itemsep=0pt}
%=============== Lstlisting Code Supp. =================
\usepackage{listings}
\usepackage{fontspec}
\usepackage{xcolor}
\definecolor{mygreen}{rgb}{0,0.6,0}
\definecolor{mygray}{rgb}{0.93,0.93,0.93}
\definecolor{mymauve}{rgb}{0.58,0,0.82}
\definecolor{myblue}{rgb}{0.76,0.8,1}
\lstset{%
frame=shadowbox,
backgroundcolor=\color{mygray},
basicstyle=\footnotesize\ttfamily,
columns=fixed,
tabsize=2,
breaklines=ture,
captionpos=b,
commentstyle=\color{blue},
escapeinside={\%*}{*)},
keywordstyle=\color{mygreen}\bfseries,
stringstyle=\color{mymauve}\ttfamily,
rulesepcolor=\color{myblue},
otherkeywords={user@machine_name\$,root@machine_name\#}
}
\lstdefinelanguage{bash}{%
morekeywords={user@machine_name\$,root@machine_name\#},
morecomment=[l]{\#},
}
%========================================================

\title{\textbf{AIRSS说明手册}}
\author{Yang Li\\ \href{mailto:lyang.1915@gmail.com}{lyang.1915@gmail.com}}
\date{2020.10.24 --- 2020.10.28}

\begin{document}
\maketitle
\tableofcontents

\newpage
\section{关于AIRSS}
AIRSS(\textbf{A}b \textbf{I}nitio \textbf{R}andom \textbf{S}tructure \textbf{S}earching)是一款由英国剑桥大学 \href{https://www.mtg.msm.cam.ac.uk/People/CJP}{Chris Pickard} 教授\footnote{Chris 教授同时还参与研发过 DFT 计算软件 CASTEP.}等人开发的第一性原理结构搜索软件. 该软件是开源的, 受\href{https://en.wikipedia.org/wiki/GNU_General_Public_License}{\textbf{ GPL2 许可证}}保护. 访问其官方网站, \url{https://www.mtg.msm.cam.ac.uk/Codes/AIRSS}, 可获取安装源码. 
  
所谓结构搜索是指: 对于一原子结构未知的体系, 在一定物理条件(如原子间距, 分布密度, 元素构成, 元素配比等)的限制下, 广泛地猜测其构型, 进行结构弛豫并计算其能量, 最终得到(DFT计算上的)全局最稳定原子结构的过程. 显然人工手动猜测或是计算机盲目地遍历式搜寻是极为笨拙、耗时、甚至难以实现的. 因此, 就需要使用一套成熟的结构搜索软件, 系统且巧妙地捕捉体系的稳定构型.
    
AIRSS 正是这样一款软件. Chris 教授本人关于此软件的介绍可参考\href{https://www.youtube.com/watch?v=xW6pOYEIKVs&t=1061s}{有关视频}. 另外还有两个较为常用的结构搜索软件: \href{http://uspex.stonybrook.edu/uspex.html}{USPEX} 和 \href{http://www.calypso.cn}{CALYPSO}.

与 USPEX 或 CALYPSO 中使用的\href{https://en.wikipedia.org/wiki/Genetic_algorithm}{遗传算法}不同, AIRSS 基于完全随机的结构搜索策略, 其产生的不同结构是完全随机且独立的, 这样的算法非常有利于搜索任务的并行实现.\footnote{比如, 我们可以用 AIRSS 随机产生1000个结构, 而后独立弛豫} 而这种随机方法的有效性在 Chris 教授关于 AIRSS 文章中也有所证明.\footnote{ [1] PRL 97, 045504 (2006); [2] JPCM 23, 053201 (2011)} AIRSS 最强大之处在于其众多灵活的可调参数, 这可以说是此项目最出众的特点. AIRSS的用户体验给人一种``傻瓜相机 vs 专业单反相机''的感觉. 前者简单实用, 可以一键提交各种不同类型的任务并获得不错的结果. 后者按键众多参数可调, 能自定义产生各种更加专业精细的结果. 前者是牺牲灵活性换取操作简易; 后者则相反. AIRSS 可以根据用户需求高度自定义研究对象的特征. 例如, 我们甚至可以使用此软件验证某个原子结构的稳定性, 或是进行晶格表面势能的计算.

令人遗憾的是, AIRSS作为一个功能强大的结构搜索软件, 其\href{https://airss-docs.github.io/technical-reference/buildcell-manual/}{官方手册}却相对简陋, 未能覆盖全部用法. 再加上网络上相关说明文档或教程十分匮乏, 所以大部分人对 AIRSS 这样一个材料设计领域异常优秀的程序并不熟悉. 

为了弥补这一缺憾, 笔者决定编写此文. 以下内容算不上指南或教程, 仅仅是学习 AIRSS 的一些记录. 本文绝大部分结论是自行分析源码、实际运行程序、尝试摸索所得.\footnote{官方虽然没有给出完整的用户手册, 但是提供了大量的使用范例, 位于程序的 example 目录中, 供使用者参考学习.} \textbf{受各种因素限制, 理解和解释上的错误或不可避免. 如有问题, 还望您不吝指正.}
\newpage
\section{准备工作}
\subsection{Linux系统} 
阅读本记录前, 您需要对Linux操作系统有一定的了解. 例如, 能理解以下指令的含义:
\begin{lstlisting}[language={bash}]
user@machine_name$ ls | grep *.cell
\end{lstlisting}

以及下述指令所能引起的灾难性事故:
\begin{lstlisting}[language={bash}]
root@machine_name# rm -rf / home/user_name/trash_directory
\end{lstlisting}

另外需要特别提醒的是, 由于PDF文档编译时会将某些字符转化为命令行不能识别的符号(如ls \(-\)1 中的减号, 虽然他们看起来与命令号输入的减号并无不同), 所以\textbf{在直接复制粘贴本文档中的命令或字符使用时, 请加以小心}.

\subsection{程序的安装与卸载}
\hyperref[sec:airss-install]{附录A} 中以 \verb|airss-0.9.0| 版本为例, 简要记录了AIRSS的安装过程. 最新的 \verb|airss-0.9.1| 在安装过程上做了大量简化, 您可以自行阅读其内包含的 \verb|README| 文件, 安装此版本.

AIRSS主运行脚本使用perl语言写成, 仅能安装在 \verb|*nix| 系统中, 且只支持在命令行(Command Line)使用. 安装此软件前, 您最好已经了解 \href{https://www.gnu.org/software/make/manual/}{GNU make} 的使用方法.

\subsection{初次运行AIRSS}
初次接触AIRSS, 您可以在终端输入 \verb|airss.pl| 指令查看软件欢迎界面.
\newpage
\begin{lstlisting}[language={bash}]
user@machine_name$ airss.pl

      .o.       ooooo ooooooooo.    .oooooo..o  .oooooo..o 
     .888.      '888' '888   'Y88. d8P'    'Y8 d8P'    'Y8 
    .8:888.      888   888   .d88' Y88bo.      Y88bo.      
   .8' '888.     888   888ooo88P'   ':Y8888o.   ':Y8888o.  
  .88ooo8888.    888   888'88b.         ':Y88b      ':Y88b 
 .8'     '888.   888   888  '88b.  oo     .d8P oo     .d8P 
o88o     o8888o o888o o888o  o888o 8::88888P'  8::88888P'  
                                                      
     Ab Initio Random Structure Searching             
     Chris J. Pickard   (cjp20@cam.ac.uk)            
            Copyright (c) 2005-2018                   
                                                      
Please cite the following:                                 
                                                      
[1] C.J. Pickard and R.J. Needs, PRL 97, 045504 (2006)     
[2] C.J. Pickard and R.J. Needs, JPCM 23, 053201 (2011)    

Usage: airss.pl [-pressure] [-build] [-pp0] [-pp3] [-gulp] 
                [-lammps] [-gap] [-psi4] [-cluster] [-slab] 
                [-dos] [-workdir] [-max] [-num] [-amp] [-mode] 
                [-minmode] [-sim] [-symm] [-nosymm] [-mpinp] 
                [-steps] [-best] [-track] [-keep] [-seed]
  -pressure f  Pressure (0.0)
  -build       Build str. only (false)
  -pp0         Use pair potentials rather than Castep (0D) (false)
  -pp3         Use pair potentials rather than Castep (3D) (false)
  -gulp        Use gulp rather than Castep (false)
  -lammps      Use LAMMPS rather than Castep (false)
  -gap         Use GAP through QUIP/QUIPPY/ASE (false)
  -ps4         Use psi4 (false)
  -vasp        Use VASP (false)
  -cluster     Use cluster settings for symmetry finder (false)
  -slab        Use slab settings (false)
  -dos         Calculate DOS at Ef (false)
  -workdir  s  Work directory ('.')
  -max      n  Maximum number of str. (1000000)
  -num      n  Number of trials (0)
  -amp      f  Amplitude of move (-1.5)
  -mode        Choose moves based on low lying vmodes (false)
  -minmode  n  Lowest mode (4)
  -sim      f  Threshold for structure similarity (0.0)
  -symm     f  Symmetrise on-the-fly (0.0)
  -nosymm   f  No symmetry (0)
  -mpinp    n  Number of cores per mpi Castep (0)
  -steps    n  Max number of geometry optimisation steps (400)
  -best        Only keep the best str. for each compos. (false)
  -track       Keep the track of good str. during RESH(false)
  -keep        Keep intermediate files (false)
  -seed     s  Seedname ('NONE')
user@machine_name$
\end{lstlisting}

\verb|airss.pl| 是使用AIRSS执行结构搜索的主要指令. 欢迎界面中已经系统且简要地说明了该指令的用法. 用法解释中的表格有三列. 第一列是传入参数的名称; 第二列是传入参数的数据类型, \verb|f|代表浮点数, \verb|n|代表整数, \verb|s|代表字符串, ``空''代表逻辑字符串 \verb!true! 或 \verb!false!. 第三列是对相应参数的简单描述.

\newpage
\section{自定义随机结构}
AIRSS 的核心组件名为 \textbf{buildcell}. 这个组件的作用就是依据用户给出的 *.cell 文件, 生成一系列结构的随机但符合给定物理条件约束的初始原子构型. 另外, 我们也可以单独拿出此模块, 使其与其他程序(如 VASP 等)适配. 学会书写出高度自定义化的 *.cell 文件是学习 AIRSS 的根本.\\


\noindent\fcolorbox{mygray}[HTML]{EEEEFF}{\parbox{\textwidth}{%
\noindent \textbf{buildcell 区块的执行大致分为以下几步:}
\begin{maineu}
\item 通过 cell.f90 模块读取 *.cell 文件中的配置信息.
\item 通过 build.f90, opt.f90 等模块生成满足要求的结构. 
  \begin{subeu}
    \item 首先根据给定的原子及其半径确定合理的晶格常数, 若存在晶格标签\footnote{如果您感觉对这一部分的概念并不熟悉, 可以先跳过本段, 更详细的介绍会在后文中展现.} \#FIX 则忽略此步骤.
    \item 按要求在随机位置生成某个原子.
    \item 根据约束选择: 接受这一位置, 拒绝这一位置, 或者\textbf{PUSH}当前原子. \textbf{PUSH}是指: 将原子位置过近的两原子在其连线方向上, 反向移开相同距离. 当原子标签中有 FIX 或 NOMOVE 时, 忽略这一操作.
    \item 遍历体系中的全部原子, 直至更新完整个结构.
    \item 视用户设置, 用唯象的对势(pp3)对结构进行简单弛豫.
  \end{subeu}
\item 通过 buildcell.f90 模块输出上述满足要求的随机结构.
\end{maineu}}}

\subsection{*.cell文件结构概括}
*.cell文件是结构搜索的``种子文件'', 您可以在此文件中设置搜索约束条件. 文件名称前的星号(*)代表是Linux系统下的通配符, 也即它代表任意字符. 在AIRSS文件系统中, 我们把后缀名前的部分称为 ``seed name''. Seed name 可用于区分不同计算的同类文件. 文中后续见到所有通配符, 如果不特殊指明, 均代指 seed name.

\noindent\fcolorbox{mygray}[HTML]{EEEEFF}{\parbox{\textwidth}{%
\noindent \textbf{*.cell 文件有以下特点:}
\begin{maineu}
\item 由于 AIRSS 是由部分参与 CASTEP 研发的人员编写, 因此该程序对 CASTEP 极其友好, 这里的 *.cell 文件与 CASTEP 中的 *.cell 文件完全兼容.
\item *.cell 文件设置参数时, 需要使用一定的 keywords 标明所设参数的含义.
\item AIRSS完全复用了CASTEP中进行结构声明的 keywords, 如 LATTICE\_CART 等.
\item AIRSS 内置的 keywords 一般以\#开头, 如\#RASH. (以使得 AIRSS 的 *.cell 文件与 CASTEP 完全兼容)
\item *.cell 文件主要由两部分组成: ``结构数据''和``全局参数''. 相应的, 文件中的 keywords 也可以分为上述两类.
\item 任何 keyword 之间没有书写顺序上的限制.
\item *.cell 中设定的所有的keywords需要采用\textbf{大写字母}书写, 并注意在keywords、等号和参数数值之间\textbf{不能添加空格}. 同时, 任何空行都将被自动忽略.
\item 该文件中使用双井号(\#\#)标明注释内容.
\item 虽然程序允许单行中出现多个 keywords, 但出于对所写文件的规范性和易读性考虑, 最好一行只写一个 keyword.
\end{maineu}}}\\

\vspace{4em}
以下是预测金属铝结构所用到的 \verb|Al.cell| 文件:
\begin{lstlisting}[language={bash},numbers=left,firstnumber=0]
user@machine_name$ cat Al.cell
%BLOCK LATTICE_CART
2 0 0
0 2 0
0 0 2 
%ENDBLOCK LATTICE_CART

%BLOCK POSITIONS_FRAC
Al 0.0 0.0 0.0 # Al1 % NUM=8
%ENDBLOCK POSITIONS_FRAC

#MINSEP=1.5
user@machine_name$
\end{lstlisting}


\noindent\fcolorbox{mygray}[HTML]{EEEEFF}{\parbox{\textwidth}{%
\noindent \textbf{上述 *.cell 文件的结构可做如下概括:}
\begin{maineu}
  \item 前9行是由 \%BLOCK [keywords] 格式定义的数据读取区块, 这种格式在CASTEP中是十分常见的. 这些区块定义了晶体基本的\textbf{结构数据}.   
  \item 1至5行数据的 [keywords] 是 LATTICE\_CART, 声明使用笛卡尔绝对坐标系定义的单胞基矢. 
  \item 7至9行使用的[keywords]是 POSITIONS\_FRAC. 这在CASTEP中的意义是``以分数坐标定义的原子位置''. 而在AIRSS中, 该数据模块不仅可以指明原子初始位置, 还可以定义搜寻过程中对单种原子的约束条件. 若初始不知道原子的具体位置, 可将其设为任意值, 如(0, 0, 0), 程序将自行找到符合要求的原子位置. 原子位置设置的更具体细节将在之后给出. 
  \item 第11行的 \#MINSEP 是 AIRSS 中的\textbf{全局参数}.
  \#MINSEP=\(1.5\)表明了任意两原子的间距不得低于\(1.5\) \r{A}.
\end{maineu}}}\\


可以看到, *.cell文件的结构大致分为两大部分:\textbf{结构数据}和\textbf{全局参数}.

\subsection{*.cell文件参数细节}
\subsubsection{结构数据}
该部分沿用了CASTEP的结构文件中定义数据的模式. 结构数据由两部分构成: 晶格参数数据区块, 原子位置数据区块. 

数据区块的具体模式如下所示:
\begin{lstlisting}[language={bash}]
%BLOCK [keywords]
[...]
[structure data]
[...]
%BLOCKEND [keywords]
\end{lstlisting}

可以在AIRSS中使用的数据区块关键字(\verb|[keywords]|)已在 \textbf{Table \ref{BLOCKkeywords}} 中列出.
\begin{center}
\begin{longtable}{m{13em}<{\centering} | m{19em}<{\centering}}
\caption{AIRSS Cheat Sheet -- Data BOLCK}
\label{BLOCKkeywords} \\
\toprule
\textbf{参数名称}  & \textbf{功能说明}  \\
\midrule
\midrule
\endfirsthead
\multicolumn{2}{c}{\tablename\ \thetable\ -- \textit{Continued from previous page}} \\
\toprule
\textbf{参数名称}  & \textbf{功能说明}  \\
\midrule
\midrule
\endhead
\midrule \multicolumn{2}{r}{\textit{Continued on next page...}} \\
\endfoot
\endlastfoot 
\verb|LATTICE_CART|  & 使用笛卡尔坐标系, 以一个\(3\times3\)的矩阵定义\textbf{单胞}基矢. 通过在该区块中附加以下标签: \verb|#FIX,|\verb|#CFIX,|\verb|#ABFIX|, 可分别实现: 固定整个晶格形状, 固定晶格常数c, 固定晶格常数ab. 另外, 需要强调的是, 如果要确保晶格参数不变, 除了在晶格参数数据区块中加入\verb|#FIX|, 以保证在猜测晶格结构时保持其为常值外, 还需要在DFT弛豫软件中做相应的设置, 以保证其在结构弛豫时也不变.\\
\midrule
\verb|LATTICE_ABC| & 以一个\(2\times3\)的矩阵定义单胞的晶格参数.矩阵第一行是参数abc, 第二行是三个夹角. 在同一个\verb|.cell|中, 与\verb|LATTICE_CART|二选一即可. 通过在该区块中附加以下标签: \verb|#FIX,|\verb|#CFIX,|\verb|#ABFIX|, 可分别实现: 固定整个晶格形状, 固定晶格常数c, 固定晶格常数ab.\\
\midrule
\verb|POSITIONS_FRAC| & 以晶格基矢坐标系下的分数坐标定义原子位置.\\
\midrule
\verb|POSITIONS_ABS| & 以笛卡尔坐标系下的绝对数值坐标定义原子位置. 与 \verb|POSITIONS_FRAC| 二者选一即可.\\
\midrule
\verb|SYMMETRY_OPS| & 定义生成的单胞所必须满足的对称操作, 四行一组, 前三行定义旋转操作, 第四行行定义平移操作. 具体用法可参考 \href{http://www.tcm.phy.cam.ac.uk/castep/documentation/WebHelp/content/modules/castep/keywords/k_symmetry_ops_castep.htm}{CASTEP:SYMMETRY OPS}.\\
\bottomrule
\end{longtable}
\end{center}

下面简单介绍, \textbf{晶格参数}数据区块和\textbf{原子位置}数据区块的详细书写方法.

\paragraph{晶格参数} 首先介绍晶格参数的设定, 以\verb|LATTICE_CART|为例.\\
\emph{例1.}
\begin{lstlisting}[language={bash},numbers=left]
%BLOCK LATTICE_CART
20 0 0
0 20 0
0 0 20
#FIX
%BLOCKEND LATTICE_CART
\end{lstlisting}

上述字段构建了一个\(20\times20\times20\) \r{A}\(^3\)的正方体作为晶体的单胞. 这里的 \verb|#FIX| 被称为\textbf{晶格标签}(Lattice Tags). 

它声明了晶格常数在搜寻(生成初始随机结构)过程中是不能改变的. 如果您想保证体系晶格常数在进行DFT弛豫时也不变, 则需要在相应计算软件的输入文件中设置.

\paragraph{原子位置} 再介绍原子位置的设置方法, 以\verb|POSITIONS_FRAC|为例.\\
\emph{例2.}
\hypertarget{Aotic Package}{\ }
\begin{lstlisting}[language={bash},numbers=left]
%BLOCK POSITIONS_FRAC
Al  0.0 0.0 0.0 # Al1 % NUM=2 
Mg  0.0 0.0 0.0 # Mg1 % NUM=4 
O   0.4 0.2 0.3 # O1  % NUM=1 POSAMP=0 FIX
O   0.1 0.1 0.1 # O2  % NUM=1 POSAMP=0 UNMOVE
H   0.3 0.3 0.6 # free_H 
H   0.0 0.0 0.0 # H-set % ANGAMP=0 POSAMP=0
H   0.0 0.0 0.0 # H-set % ANGAMP=0 POSAMP=0
%BLOCKEND POSITIONS_FRAC
\end{lstlisting}

通过这个例子可以看出, 原子位置结构数据的基本格式是:
\begin{lstlisting}[language={bash}]
[element] [x] [y] [z] # [atoms_set_name] % [tag1] [tag2] [tag3]
\end{lstlisting}

\noindent\fcolorbox{mygray}[HTML]{EEEEFF}{\parbox{\textwidth}{%
\noindent \textbf{每一行内部的元素含义如下: }
\begin{maineu}
  \item 第一项是元素名称, 如果\textbf{将元素名称设置为Z}, 则表示此原子是个空位, 空位与真实原子一样具有体积和PUSH属性, 但不会在最终结果中输出.
  \item 二三四项是原子位置坐标.
  \item 井号(\#)后的第一项是原子所在原子集的名称, 原子集名称可以被设置成任意字符. \textbf{原子集名称相同的原子组成一个原子集, 原子集是由一个或多个原子构成的基本结构单元, 同一个原子集中的原子做相同的随机移动, 且不受PUSH影响}. 如无特殊需求, 为了实现自由度相对最大化, 一般不推荐将不同原子放入同一原子集中.
  \item 百分号(\%)之后的内容都是\textbf{原子标签}(Atom Tags). 同一行原子可以指定多个原子标签, 中间用空格隔开, 他们共同指定了该原子应该满足的若干约束条件.
\end{maineu}}}\\

\textbf{Table \ref{AIRSS_Atom_Tag}} 是AIRSS中全部可用的原子标签的具体说明.

\begin{center}
\begin{longtable}{m{10em}|m{4em}<{\centering}|m{3em}<{\centering}|m{15em}}
\caption{AIRSS Cheat Sheet -- Atom Tags (distance in \r{A})}
\label{AIRSS_Atom_Tag}\\
\toprule
\textbf{参数名称} & \textbf{输入类型} & \textbf{默认值} & \textbf{功能说明}  \\
\midrule
\midrule
\endfirsthead
\multicolumn{4}{c}{\tablename\ \thetable\ -- \textit{Continued from previous page}} \\
\toprule
\textbf{参数名称} & \textbf{输入类型} & \textbf{默认值} & \textbf{功能说明}  \\
\midrule
\midrule
\endhead
\midrule \multicolumn{4}{r}{\textit{Continued on next page...}} \\
\endfoot
\endlastfoot
\verb|NUM=|\(n\)\hspace{4em} \verb|NUM=|\(n_{min}-n_{max}\) & \verb|int| & 1 & 定义该行原子实际代表的原子个数. 使用 \verb|NUM| 定义的重复原子会被逐个单独分配到不同的\textbf{原子集}中.\\
\midrule
\verb|POSAMP=|\(d\)& \verb|float| & -1.0 & 定义该行原子在结构搜索过程中, 能偏离初始位置的最大距离. 若为负值则无限制.\\
\midrule
\verb|MINAMP=|\(d\)& \verb|float| & 0.0 & 定义该行原子在结构搜索过程中, 偏离初始位置的最小距离.\\
\midrule
\verb|XAMP=|\(d\)& \verb|float| & -1.0 & 定义该行原子在结构搜索过程中, 在X轴方向上移动的最大距离. 若为负值则无限制. 设置该项会令POSAMP和MINAMP失效\\
\midrule
\verb|YAMP=|\(d\)& \verb|float| & -1.0 & 定义该行原子在结构搜索过程中, 在Y轴方向上移动的最大距离. 若为负值则无限制. 设置该项会令POSAMP和MINAMP失效.\\
\midrule
\verb|ZAMP=|\(d\)& \verb|float| & -1.0 & 定义该行原子在结构搜索过程中, 在Z轴方向上移动的最大距离. 若为负值则无限制. 设置该项会令POSAMP和MINAMP失效.\\
\midrule
\verb|ANGAMP=|\(\theta\)  & \verb|float| \([0,360]\) & -1.0 & 原子绕自身所在\textbf{原子集}中心旋转角度的最大值(与 \verb|POSAMP| 互相独立, 分开作用). 存在晶格标签 \verb|#FIX| 时, \verb|ANGAMP=0|, \verb|POSAMP=0| 与 \verb|NOMOVE(\FIX)| 共用, 可保证原子在原始位置不动. 如果cell文件中所有原子集都只有1个原子, 则该参数失效, \verb|ANGAMP| 强制设为0. 若为负值则无限制.\\
\midrule
\verb|RAD=|\(d\) & \verb|float| & 0.0 & 设置原子的半径, 用于判断两个原子间的最小距离, MINSEP. 其优先级低于直接设置全局参数 \verb|#MINSEP=|.\\
\midrule
\verb|FIX| & \verb|void| & off & 该原子在产生结构时不受PUSH影响. 任何接近该原子的可移动原子都会以2倍的PUSH步长被PUSH回去. 同时向cell文件中写入指令使其在CASTEP弛豫时也不动. 这一指令仅在晶格被 \verb|#FIX| 时有效. 原子标签中的 \verb|FIX| 与晶格标签中的 \verb|#FIX|, \verb|#CFIX|, \verb|#ABFIX|名称相似, 但作用不同. \\
\midrule
\verb|NOMOVE| & \verb|void| & off & 该原子在产生结构时不受PUSH影响, 任何接近该原子的可移动原子都会以2倍的PUSH步长被PUSH回去. 这一指令仅在晶格被 \verb|#FIX| 时有效.\\
\midrule
\verb|COORD=|\(n\),\hspace{5em} \verb|COORD=|\(n_{min}-n_{max}\) & \verb|int| & -1 & 约束该行原子的配位数(最近邻原子数). 若为负值则无限制.\\
\midrule
\verb|NN=|\(\pm{}elm\)  & \verb|string| & null & 规定原子最近邻元素种类, `\(+\)'代表必须近邻该元素, `\(-\)'代表不能近邻该元素. 一个原子只能指定一种元素近邻或不近邻(NN参数仅会被读入一次). 若为空, 则表示无限制\\
\midrule
\verb|OCC=|\(p\) & \verb|float| \([0,1]\) & 1.0 & 该点位占据原子的几率, 同种元素在不同位置的占据几率之和必须为大于0的整数. 接收分数形式的输入, 如 \verb|OCC=1/3|. 若该值设为小于0的数, 则强行将该原子的OCC设置为1, 同时不再输出该原子可能因对称性衍生的其他原子位置. 若原子同时有 \verb|FIX| 或 \verb|NOMOVE| 标签, 则 \verb|OCC| 强制设为1. 若该原子OCC大于1, 则将其设置为空位(与将元素名称设为Z的效果相同). 系统存在对称性时, 使用更小的 OCC 可强制某一原子位于对称性更高的点位上. 但更为自然的用法是使用 \verb|MULT| 代替 \verb|OCC| 设置. 若设置了全局参数\verb|#NFORM=n|(n大于0): 在没有对称性条件下, \verb|OCC| 参数失效; 在有对称性条件下, 请参考 \verb|MULT| 中的说明. \\
\midrule
\verb|MULT=|\(m\) & \verb|float| & -1.0 & 设置该原子点位的 multiplicities. 这一数值为正时, 强制将 \verb|OCC| 参数的值覆盖为 MULT/SYMM\_NUM. 其中 SYMM\_NUM 为晶格对称群元的个数. 若设置了全局参数\verb|#NFORM=n|(n大于0), 则等效于将所有原子的 \verb|MULT| 的值设为\(n\), 同时不设置\verb|#NFORM=|. 若该行原子有标签 \verb|FIX| 或 \verb|NOMOVE|, 则强制将该原子的MULT设为-1. 使用这一参数, 可以限制某一行原子由晶体对称性实际生成原子的个数, 以及所处位置对称性的高低. 关于格点 multiplicities 与对称性的关系可参考 \href{https://www.cryst.ehu.es/cryst/get_set.html}{BCS: Wyckoff Position}.\\
\midrule
\verb|PERM| & \verb|void| & off & 在完成原子位置选定后, 重排(按一定概率互相交换)指定原子位置间的元素种类. 需要配合全局参数 \verb|#PERMUTE| 使用, 否则该参数将被关闭.\\
\midrule
\verb|ADATOM| & \verb|void| & off & 表明该行原子是等, 未加该标签的所有原子位置生成结束之后, 再加入的原子.\\
\midrule
\verb|ATHOLE| & \verb|void| & off & 表明该原子位于被挖掉的孔洞之中, 输出结构中将自动删除该原子. 常与全局变量 \verb|#HOLE=|, \verb|#HOLEPOS=| 联用.\\
\bottomrule
\end{longtable}
\end{center}

\subsubsection{全局参数}

结构搜索过程中整个体系应遵守的条件由全局参数指定.  AIRSS中的全局参数均以井号(\verb|#|)开头, 对体系中晶格以及全部的原子作用. 另外, 有些全局参数和原子标签有相同的作用, 此时, 原子标签设置的优先级要高于全局参数.\\

\noindent\fcolorbox{mygray}[HTML]{EEEEFF}{\parbox{\textwidth}{%
\noindent \textbf{全局参数目前覆盖的功能大致可以分为以下几类:}
\begin{maineu}
  \item \textbf{对晶胞参数的约束.} 比如固定初始晶胞, 约束晶胞体系, 约束晶胞变化程度, 约束晶胞所在晶系, 为晶胞添加真空层等.
  \item \textbf{对体系化学式的要求.} 比如体系是几元的, 电子是否能配平(是否有悬挂键)等.
  \item \textbf{对原子位置的约束.} 如两原子之间的最短距离, 某个原子偏离初始位置的最大/最小距离, 某原子的配位数等. 
  \item \textbf{对体系对称性的约束.} 如体系有几个对称操作, 在什么空间群, 对称性寻找的精细程度, 是否引入破坏对称性扰动等.
  \item \textbf{对程序本身搜索算法的调整.} 如, 接受搜索失败次数上限, 是否引入PSUH, 是否引入TPSD, 是否引入RASH等.
  \item (不太常用的) 在晶体中加入一个覆盖整个晶胞的力场. 如, 加入一个球形力场, 椭球力场, 长条状力场, 平面力场等.
\end{maineu}}}\\


AIRSS中全部可用的全局参数的功能和详细使用方法如\textbf{Table \ref{AIRSS_Cheat_Sheet_Pragma}} 所示.

\begin{center}
\begin{longtable}{m{11em}|m{4em}<{\centering}|m{3em}<{\centering}|m{15em}}
\caption{AIRSS Cheat Sheet -- Global Pragma (distance in \r{A})}
\label{AIRSS_Cheat_Sheet_Pragma} \\ 
\toprule
\textbf{参数名称} & \textbf{输入类型} & \textbf{默认值} & \textbf{功能说明}  \\
\midrule
\midrule
\endfirsthead
\multicolumn{4}{c}{\tablename\ \thetable\ -- \textit{Continued from previous page}} \\
\toprule
\textbf{参数名称} & \textbf{输入类型} & \textbf{默认值} & \textbf{功能说明}  \\
\midrule
\midrule
\endhead
\midrule \multicolumn{4}{r}{\textit{Continued on next page...}} \\
\endfoot
\endlastfoot
\verb|#CELLCON=|\(a\;b\;c\;\alpha\;\beta\;\gamma\) & \verb|int| & off &定义晶胞应满足的条件: \(a,b,c\)是晶格常数; \(\alpha,\beta,\gamma\)是三个晶角. 晶格常数可选填的值有\(\{0,-1\}\), 晶角可选填的值有\(\{0,-1, \theta\}\). \(0\)表示无约束, \(-1\)表示相等. 如 \verb|#CELLCON=-1-1 -1 90 90 90| 表示立方晶系.\\
\midrule
\verb|#SYSTEM=|\(sys\) & \verb|string| & off & 设置结构所在的晶系, 可选的输入有 [Tric, Mono, Hexa, Rhom(/Trig), Orth, Tetr, Cubi]. 该参数实际为 \verb|'#CELLCON='| 参数的若干特殊组合.\\
\midrule
\verb|#CONS=|\(p\) & \verb|float| \((0,1]\) & 0.4 & 约束单胞边长. 接近0表示完全没有约束, 等于1表示尽一切可能约束晶胞abc三边等长.\\
\midrule
\verb|#ACONS=|\(p\)& \verb|float| \((0,1]\) & 0.5 & 约束晶胞晶角, 使其远离某一晶格参数极小、晶胞扁平的情况. 接近0表示完全没有约束, 等于1表示尽一切可能约束三个键角严格等于90度.\\
\midrule
\verb|#CELLAMP=|\(p\)& \verb|float| & -1.0 & 在一定限度内, 随机变化晶格形状. 设为负值时采用晶格形状将完全随机变化. \(p\)越小, 晶格偏离cell文件中给定值越小. 若要启用该参数, 建议不要设置超过1.0的值.\footnote{该参数使用方法详见 airss-0.9.1/src/buildcell/src/build.f90 第100行cellamp变量的使用.} 使用该参数后, 晶格标签 \verb|#ABFIX| 与 \verb|#CFIX|, 原子标签 \verb|#CONS| 与 \verb|#ACONS| 都将被弃用. 若为负数则关闭该参数设置.\\
\midrule
\verb|#VACUUM=|\(d\) & \verb|float| & 0.0 & 在晶格(Z方向)中加入\(d\) \r{A}的真空层.\\
\midrule
\verb|#NFORM=|\(n\) & \verb|int| & -1 & 声明单胞中实际存在的原子个数是的 \verb|%BLOCK POSITIONS_*| 或 \verb|#SPECIES| 中定义的\(n\)倍. 该选项不可与 \verb|#NATOM| 联用. 若为负值则此参数被关闭.\\
\midrule
\verb|#SUPERCELL=|\(n\) \verb|#SUPERCELL=|\(n_a\;n_b\;n_c\;\;\;\;\;\;\) \hspace{2em}\verb|#SUPERCELL=|\(a_x\; a_y\;a_z\;\)\hspace{2em} \(b_x\;b_y\;b_z\;c_x\;c_y\;c_z\) & \verb|int| & off & 定义超胞的尺寸. 可以使用超胞中单胞的的个数\(n\), 超胞晶格基矢在三个方向的数值\(n_a\;n_b\;n_c\), 或是超胞与单胞晶格基矢之间的变换矩阵\([[a_x,a_y,a_z],[b_x,b_y,b_z],[c_x,c_y,c_z]]\)定义新的超胞. 超胞的构建策略是: 首先用单胞构建符合条件的原子构型, 而后根据扩胞参数复制粘贴得到超胞后直接输出.\\
\midrule
\verb|#SLAB| & \verb|void| & off & 声明该结构是一个二维平面结构, 取超胞时将不再向c方向扩胞, 同时检查结构对称性时也将忽略c方向.\\
\midrule
\verb|#SURFACE| & \verb|void| & off & 声明该结构是一个表面结构, 表面结构继承SLAB的全部特性, 同时增加一条与表面相关的对称性限制条件.\footnote{见源码airss-0.9.1/src/buildcell/src/cell.90 第3216行}\\
\midrule
\verb|#MOLECULES|& \verb|void| & off & 声明同一个原子集中的原子构成了一个基本分子构型, 检查配位时将不再对同一个原子集内的原子做检查.\\
\midrule
\verb|#CLUSTER| & \verb|void| & off & 声明预测的结构是无周期性的团簇.\\
\midrule
\verb|#FOCUS=|\(n\) & \verb|int| & 0 & 约束最终结构必须是\(n\)组分的(由\(n\)种不同的元素构成). 该参数一般用于变组分分析. 小于等于0表示无约束.\\
\midrule
\verb|#OCTET|& \verb|void| & off & 检查化学配比是否电子配平(全部电子数是否可以被8整除).\\
\midrule
\verb|#POSAMP=|\(d\) & \verb|float|  & -1.0 & 与原子标签中的定义相同, 结构搜索中原子偏离初始点的最大位置. 负值为无限制.\\
\midrule
\verb|#MINAMP=|\(d\) & \verb|float| & -1.0 & 与原子标签中的定义相同, 结构搜索中原子偏离初始点的最小位置. 负值为无限制.\\
\midrule
\verb|#XAMP=|\(d\) & \verb|float| & -1.0 & 与原子标签中的定义相同, 结构搜索中原子X方向偏离初始点的最大振幅. 负值为无限制. 设置该项会令 POSAMP 和 MINAMP 失效. \\
\midrule
\verb|#YAMP=|\(d\) & \verb|float| & -1.0 & 与原子标签中的定义相同, 结构搜索中原子Y方向偏离初始点的最大振幅. 负值为无限制. 设置该项会令 POSAMP 和 MINAMP 失效.\\
\midrule
\verb|#ZAMP=|\(d\) & \verb|float| & -1.0 & 与原子标签中的定义相同, 结构搜索中原子Z方向偏离初始点的最大振幅. 负值为无限制. 设置该项会令 POSAMP 和 MINAMP 失效.\\
\midrule
\verb|#ANGAMP=|\(\theta\)& \verb|float| \([0,360]\) & -1.0 & 与原子标签中的定义相同, 原子绕自身所在\textbf{原子集}中心旋转角度的最大值. 负值为无限制.\\
\midrule
\verb|#MINBANGLE=|\(\theta\) & \verb|float| \((0,360]\) & 0.0 & 设置搜索结构中, 原子键角的最小值.\\
\midrule
\verb|#MAXBANGLE=|\(\theta\) & \verb|float| \((0,360]\) & 180.0 &设置搜索结构中, 原子键角的最大值.\\
\midrule
\verb|#MINSEP=|\(d\) \verb|#MINSEP=|\(d\;\)\verb|X-X=|\(d_{X−X}\;\)\ \ \  \ \verb|X-Y=|\(d_{X−Y}\;...\) & \verb|float| & 0.0 & 两原子间最小距离, 也可以用来定义两原子距离固定是多少. 比如 \verb|#MINSEP=| \verb|2.0 Li-Li=2.6 Ge-Ge=2.51|\\
\midrule
\verb|#RAD=|\(d\)& \verb|float|  & 0.0 & 与原子标签中的定义相同, 定义原子的半径大小.\\
\midrule
\verb|#COORD=|\(n\)& \verb|int| & -1 & 与原子标签中的定义相同, 设置原子的配位数限制. 负值为无限制.\\
\midrule
\verb|#SPIN=|\(S_{real}\;S_{mod}\;\;\;\;\;\;\) \verb|#SPIN=|\(S_{real}\;S_{mod}\;\;\;\;\;\) \verb|"|elm\(_1\) elm\(_2\)... \verb|"| & \verb|float|,  \verb|string|& off & 随机设置体系每个原子上共线自旋的取值. 并要求 \(S_{t,real}\)/\(N_{ions}\) = \(S_{real}\), \(S_{t,mod}\)/\(N_{ions}\) = \(S_{mod}\), 其中\(S_{t,real}\)是全部指定原子自旋的和, \(S_{t,mod}\)是全部指定原子自旋绝对值的和. 未被指定的原子自旋保持为0. 可以用被双引号包裹的带空格的元素符号指明哪些是指定原子. 若未指定任何原子, 则默认所有原子都是指定原子. 例如 \verb|#SPIN=0 5 "Fe Co "| 或者 \verb|#SPIN=1 4| \\
\midrule
\verb|#SPECIES=|\(elm_1\%tags_1,\) \(elm_2\%tags_2,...\;\;\;\;\;\;\) \verb|#SPECIES=|\(elm_1,elm_2...\) & \verb|string|  & off & 使用简化记号定义体系原子组分. 例如, \verb|#SPECIES=Si|\verb|%NUM=1| \verb|COORD=2,|\verb|O%NUM=2|. 该参数不能与\verb|BLOCK POSITIIONS_*|同时出现.\\
\midrule
\verb|#NATOM=|\(n\) \hspace{6em}\verb|#NATOM=|\(n_{min}-n_{max}\)  & 0 & \verb|int| & 与 \verb|#SPECIES| 联用, 定义单胞中总原子数, 一般用于变组分分析(变胞预测). 且使用此全局参数会使 \verb|#SPECIES| 指令包含的\verb|%|后的原子标签全部失效. \footnote{详细原因请参见源码``airss-0.9.1/src/buildcell/src/cell.f90''第493行与514行区别.} 若 \verb|#SPECIES| 中含有多中元素, 则每种元素随机数目, 保持总和为NATOM. 该参数设为0时自动失效.\\
\midrule
\verb|#TARGVOL=|\(V\) \verb|#TARGVOL=|\(V_{min}-V_{max}\) & \verb|float| & init. cell vol. & 固定晶格体积为\(V\), 或\(V_{min}\)到\(V_{max}\)之间的随机数值. 存在晶格标签 \verb|#FIX| 时该参数失效. 默认为初始给定原胞的体积. 该参数常在不引入 \verb|BLOCK POSITION_*| 区块时使用.\\
\midrule
\verb|#VARVOL=|\(V\)& \verb|float| & init. cell vol. & 与 \verb|#TARGVOL=| 作用相同, 该参数会覆盖 \verb|#TARGVOL=| 的设置. 默认为初始给定原胞的体积.\\
\midrule
\verb|#SLACK=|\(p\)& \verb|float| \([0,1)\) & 0.0 & 使用此参数可整体降低体系对原子间成键(原子相对位置)的限制, \(p\)越大对原子间间距和角度的要求越低. 默认为0, 推荐值0.1--0.3.\\
\midrule
\verb|#AUTOSLACK=|\(p\)& \verb|float| \([0,1)\) & off & 使用此参数可整体降低体系对原子间成键(原子相对位置)的限制, \(p\)越大对原子间间距和角度的要求越低. 若给定的初始值\(p\)不合适, 则以0.01的步长递增SLACK, 直到找到合适的SLACK值.\\
\midrule
\verb|#FLIP|& \verb|void| & off & 搜索结构时, 对\textbf{原子集}引入随机翻转操作. 若体系中所有原子集中均只有一个原子, 则该参数失效.\\
\midrule
\verb|#REMOVE|& \verb|void| & off & 删除(PUSH后)重叠的原子(之一). 可以用于高初始原子密度的结构预测.\\
\midrule
\verb|#TIGHT|& \verb|void| & off & 使得生成的结构更加紧密.\\
\midrule
\verb|#SYMMOPS=|\(n\)\hspace{6em}\verb|#SYMMOPS=~|\(n\;\;\;\;\;\;\;\;\;\;\;\)\hspace{6em}\verb|#SYMMOPS=|\(n_{min}-n_{max}\) & \verb|int| &  off & 声明生成的结构中必须含有\(n\)种对称操作. 若体系是周期性晶体结构, 推荐从下述整数中选取: 1,2,3,4,5,6,8,12,16,24,48. 若输入中含有波浪号(\(\sim\))则表明, 结构搜索时将只在general positions上放置原子, 对于对称性更高但数量\footnote{这里的``数量''是指, 满足对称操作所需的最少同类原子数量, 也即该点位的 multiplicities. }更少的 special position 不予考虑.\footnote{源代码中这样描述: Symmetry is only be approximately applied (filling general positions only)} 另外需要注意的是, AIRSS中实现锁定对称性的方法是: 接将原子按对称性复制n份. 例如原子有\(C_4\)对称性, 则最终输出的总原子数将是设定的4倍. 但如果结合原子标签 \verb|OCC, MULT| 等的使用, 可实现在更高对称的点位安放更少原子的操作. 在含有波浪号(\(\sim\))的模式下不推荐使用\verb|MULT|等参数限制原子对称复制的个数.\\
\midrule
\verb|#SYMM=|\(spg\)\hspace{6em} \verb|#SYMM=~|\(spg\)& \verb|string| & off & 生成的结构必须在\(spg\)空间群中, \(spg\)是空间群的名称. 若输入中含有波浪号(\(\sim\))则表明, 结构搜索时将只在general positions上放置原子, 对于对称性更高但数量更少的 special position 不予考虑, 在此模式下不推荐使用\verb|MULT|等参数限制原子对称复制的个数.\\
\midrule
\verb|#SYMMNO=|\(n\)\hspace{6em} \verb|#SYMMNO=~|\(n\) & \verb|int| & off & 生成的结构必须在第\(n\)号空间群中, \(n\)是空间群的序号. 若输入中含有波浪号(\(\sim\))则表明, 结构搜索时将只在general positions上放置原子, 对于对称性更高但数量更少的 special position 不予考虑, 在此模式下不推荐使用\verb|MULT|等参数限制原子对称复制的个数.\\
\midrule
\verb|#SYMMORPHIC| & \verb|void| & off & 只检查体系是否存在点式对称操作.\\
\midrule
\verb|#SGRANK=|\(n\) & \verb|int| & 230 & 设置空间群寻找/锁定的序号上限\(n\). 此值设为230时, 接受任何空间群的对称性锁定.\\
\midrule
\verb|#ADJGEN=|\(n\)\hspace{4em}\verb|#ADJGEN=|\(n_{min}-n_{max}\)  & \verb|int|  & 0 & 调整晶胞中使用 general positions(GP) 的个数. 该值为0时, 会最大程度地使用GP点位. 增大该值则将逐渐更多地使用 Special positions(SP)点位. 如果尝试后发现难以生成符合条件结构, 则程序将动态增加该值. 关于 SP/GP 与晶体空间群的关系请参考: \href{https://en.wikipedia.org/wiki/Wyckoff_positions}{Wiki: Wyckoff Position}.\\
\midrule
\verb|#BREAKAMP=|\(d\) & \verb|float| & 0.0 & 在晶格矢量a方向随机移动原子破坏原有对称性, 原子分数坐标移动距离: \(d^{(frac)}_a=(random(0,1)\times{}d)^{1/3}\)\\
\midrule
\verb|#NOPUSH|& \verb|void| & off & 对于距离过近的原子不引入PUSH步, 直接拒绝该构型. 该关键字不会关闭 \verb|#SPHERE| 等关键字引入的晶体中心势场的PUSH步.\\
\midrule
\verb|#PUSHSTEP=|\(p\)& \verb|float| & 0.25 & 每一步PUSH移动距离大小(stepsize)的比例参数.\\
\midrule
\verb|#PUSHMAX=|\(n\)& \verb|int| & 100 &设置最大PUSH步数(在 \verb|buildcell| 的输出中, 两个X之间所夹的 \verb!* | : -! 符号的个数).\\
\midrule
\verb|#OVERLAP=|\(cov\)& \verb|float| & off & 结束PUSH之后, 再附加采用TPSD\footnote{Two-point Step Size Gradient Methods - Barzilai and Borwein, IMA Journal of Numerical Analysis (1988) 8, 141-148}对势对晶体结构进行简单弛豫. \(cov\)为收敛判据, 其值越小对晶格收敛限制越高, 推荐值为0.1--0.2.\\
\midrule
\verb|#RASH|& \verb|float| & off & 在使用TPSD对势弛豫结束后再引入原子集之间的随机位移和旋转(SHAKE step). 设置过 \verb|#OVERLAP| 之后此参数才有效.\\
\midrule
\verb|#RASH_POSAMP=|\(d\)& \verb|float| & 1.0 & 设置由 RASH 引入的 SHAKE 步移动原子集的最大距离, 与POSAMP类似. 该项必须为一个小的正值, \textbf{不可设为负数}.\\
\midrule
\verb|#RASH_ANGAMP=|\(\theta\) & \verb|float| \((0,360]\) & 30.0 & 设置由 RASH 引入的 SHAKE 步原子集绕自身中点转动的最大角度, 与ANGAMP类似. 该项必须为一个小的正值, \textbf{不可设为负数}.\\
\midrule
\verb|#CELLADAPT|& \verb|void| & off & 在设置\verb|#OVERLAP|的情况下附加设置此项, 可强制要求TPSD简单结构弛豫时同时尝试在保持体积不变的情况下改变单胞的形状. 体系默认不会在TPSD结构优化步改变单胞形状. 若存在晶格标记\verb|#FIX| 或者设置了 \verb|#CELLAMP=0| 则此参数失效. \\
\midrule
\verb|#THREE=|\(p\)& \verb|float| & TODO & 使用三体势能代替TPSD弛豫结构, \textbf{该参数对应的功能尚未在airss-0.9.1中实现}.\\
\midrule
\verb|#COMPACT|& \verb|void| & on & 对最终生成的单胞进行 niggli reduce 操作. 在晶格形状没有被锁定时(未引入晶格标签\verb|#FIX, #CFIX, #ABFIX|), 该项默认打开. \href{http://atztogo.github.io/niggli/#algorithm}{R. W. Grosse-Kunstleve, N. K. Sauter and P. D. Adams, Acta Cryst., A60, 1-6 (2004)} \\
\midrule
\verb|#NOCOMPACT| & \verb|void| & off & 强制关闭 COMPACT 操作.\\
\midrule
\verb|#PERMUTE| & \verb|void| & off & 在完成原子位置选定后, 重排(按一定概率互相交换)指定原子位置间的元素种类. 可以联合原子标签 \verb|PERM| 使用.\\
\midrule
\verb|#PERMFRAC=|\(p\) & \verb|float| \([0,1]\)  & 1.0 & 设置重排发生的概率.\\
\midrule
\verb|#HOLE=|\(d\) & \verb|float| & -1.0 & 设置在晶格上切割球洞的半径. 设为负数时不对成型的结构做任何处理.\\
\midrule
\verb|#HOLEPOS=|\(f_a\;f_b\;f_c\) & \verb|float| & random & 设置在晶格上切割球洞的位置(分数坐标). 默认为随机位置. 可与原子标签中的\verb|ATHOLE|联用.\\
\midrule
\verb|#VACANCIES=|\(n\)\verb|@|\(elm\)& \verb|float|, \verb|string| & off & 等待结构生成完毕后, 选取\(n\)个元素种类为elm的原子替换为空位. \footnote{该参数本来还设计了一种不加@符号的输入, 但目前在airss-0.9.1中, 这种输入在后续处理中存在一些BUG, 因此在此未予说明.}\\
\midrule
\verb|#MAXTIME=|\(t\)& \verb|float| & 1.0 & 设置对一个猜测结构PUSH步使用时间的上限, 超过该时间程序将停止PUSH并重新猜测新结构. 默认\(t=1\)s.\\
\midrule
\verb|#NFAILS=|\(n\)& \verb|int| & 0 & 每个结构允许的失败次数(在 \verb|buildcell| 命令输出中出现 \verb|X| 的次数). 若其值为0, 则无限制.\\
\midrule
\verb|#SPHERE=|\(r\) & \verb|float| & off & 在单胞中心处引入一球状势能. 设置球势能的吸引半径为\(r\). 当原子与晶格中心距离大于\(r\)时, 会受到指向晶格中心的PUSH.\\
\midrule
\verb|#ELLIPSOID=|\(r\;\varepsilon\) & \verb|float|  & off & 在单胞中心处引入一椭球状吸引势. \(r\) 为椭球势能半长轴长度, \(\varepsilon\)为形变程度. 当原子与晶格中心距离大于\(r\)时, 会受到指向晶格中心的PUSH. \(\varepsilon=0\)为球形, \(\varepsilon\)越大畸变越严重. \\
\midrule
\verb|#PANCAKE=|\(r\;\varepsilon\) & \verb|float|  & off & 在单胞中心处引入一圆饼状吸引势. \(r\)为圆饼半径, \(\varepsilon\)为形变程度. 当原子与晶格中心距离大于\(r\)时, 会受到指向晶格中心的PUSH. \(\varepsilon=0\)为偏平圆饼, \(\varepsilon=1\)为接近球形的势能. \\
\midrule
\verb|#CIGAR=|\(r\;\varepsilon\) & \verb|float| & off & 在单胞中心处引入一雪茄状吸引势. \(r\)为雪茄长度, \(\varepsilon\)为形变程度. 当原子与晶格中心距离大于\(r\)时, 会受到指向晶格中心的PUSH. \(\varepsilon=0\)为针尖状势能, \(\varepsilon=1\)为接近球形的势能. \\
\midrule
\verb|#CYLINDER=|\(r\) & \verb|float| & off & 在单胞中心处引入一圆筒状势能(原子在Z方向不受力). \(r\)为圆筒势能吸引半径. 当原子与晶格中心距离大于\(r\)时, 会受到指向晶格中心的XY方向的PUSH.\\
\midrule
\verb|#CORE=|\(r\) & \verb|float| & off & 对定义的球状(椭球状, 圆饼状, 雪茄状, 圆筒状)吸引势附加排斥核心. 设置排斥核心半径(长轴长度, 半径, 雪茄长度, 圆筒半径)为\(r\). 当原子与晶格中心距离小于\(r\)时, 会受到远离晶格中心方向的PUSH.\\
\midrule
\verb|#WIDTH=|\(l\)& \verb|float| & off & 使用平面状势能(原子在X和Y方向均不受力), 附加计算PUSH步的距离. \(l\)为平面状势能吸引长度. 当原子与原点(origin)距离大于\(l\)时, 会受到指向原点的Z方向的PUSH.\\
\midrule
\verb|#SHIFT=|\(h\)& \verb|float|& 0.0  & 将平面势移动至Z=\(h\)的位置(origin的位置), 默认\(h=0\).\\
\bottomrule
\end{longtable}
\end{center}

\newpage
现在您应该可以轻松读懂下述内容了:

\begin{lstlisting}[language={bash},numbers=left]
%BLOCK LATTICE_CART
20 0 0
0 20 0
0 0 20
#FIX
%ENDBLOCK LATTICE_CART

%BLOCK POSITIONS_FRAC
Al 0.0 0.0 0.0 # Al1 % NUM=7-13 COORD=4
%ENDBLOCK POSITIONS_FRAC

#MINSEP=1.5
#CLUSTER
#OVERLAP=0.2
#RASH
#POSAMP=3.0
#MINBANGLE=80
#MAXBANGLE=120
\end{lstlisting}

另外, *.cell 文件中还可以完全不出现结构数据对应的数据区块. 比如您只对晶体大概的结构有一些模糊的认识(比如, 只知道其晶体原子构成, 晶格大概的体积大小等), 仍然可以使用AIRSS进行结构搜索. 下面就是这样两则符合规范且十分简洁的AIRSS结构种子文件.\\
\emph{例4.1}
\begin{lstlisting}[language={bash},numbers=left]
#VARVOL=15
#SPECIES=A%NUM=4,B%NUM=1
#NFORM=2 
#MINSEP=1.5
\end{lstlisting}
\emph{例4.2}
\begin{lstlisting}[language={bash},numbers=left]
#VARVOL=15 
#SPECIES=A,B,C
#NATOM=2-8
#MINSEP=1.5        
\end{lstlisting}

这里用 \verb|#VARVOL| 代替了晶格参数数据区块 \verb|BLOCK LATTICE_CART|, 用 \verb|#SPECIES| 代替了原子位置数据区块 \verb|BLOCK POSITIONS_FRAC|.

\newpage
\section{结构弛豫与能量计算}

\subsection{联合airss-pp3模块弛豫}
使用 AIRSS 联合 \verb|airss-pp3| 计算模块预测结构时, 除了要准备一个 *.cell 的文件外, 还需要准备一个名为 *.pp 文件. 

\verb|airss-pp3|是AIRSS自带的pp3对势(pair potential)计算模块. 其功能是使用化学上经典的分子势场(如6-12势)弛豫原子结构并输出体系能量. 由于这一模块使用的是维象的晶体能量模型, 不涉及任何第一性原理的复杂运算, 因此其实现简单、效果稳定、计算速度极快, 适用于简单组分的小体系. 更具体的, \verb|airss-pp3| 采用了如下势能计算原子的受力和体系总能:
\begin{subequations}
  \label{eq:pp3}
  \begin{align}
    E_{ij} &= 4\varepsilon_{ij}\left[\left(\dfrac{\sigma_{ij}}{r_{ij}}\right)^m -\beta_{ij}\left(\dfrac{\sigma_{ij}}{r_{ij}}\right)^n\right]\\
    E &= \sum_{i<j}^{\text{all ions}} E_{ij}
  \end{align}
\end{subequations}
其中 \(i,j\)标定了不同原子位置, 当元素种类不同时, \(\varepsilon_{ij},\sigma_{ij}\)等变量对应不同的值.

使用该模块时, 需要首先配置名为*.pp的参数文件. 这个文件中存储了对势的相关参数, 其内部书写形式如下: 
\begin{lstlisting}[language={bash},numbers=left]
n_spec m n range
specs
# Epsilon
eps_11 eps_12 ...
eps_22 ...
...
# Sigma
sgm_11 sgm_11 ... 
sgm_11 ... 
...
# Beta
beta_11 beta_12 ...
beta_22 ...
...
\end{lstlisting}

\noindent\fcolorbox{mygray}[HTML]{EEEEFF}{\parbox{\textwidth}{%
\noindent \textbf{上述参数中, }
\begin{maineu}
  \item 第1行各项分别是: 元素个数; 对势中指数\(m,n\)的数值; 对势中能量极小值(力为0处)处距离原子中心的距离\(d_{min}\)与\(\sigma\)的比值, 也即 \(d_{min}^{(ij)} = \sigma_{ij}*d_{range}\). 
  \item 第2行指明了体系中的元素种类, 不同元素间用空格隔开. 
  \item 第3行是注释行, 在程序中无意义, 但必须存在. 
  \item 第4至(\(n_{spec}+3\))行声明了元素之间的\(varepsilon\)值对应的矩阵.
  \item 第(\(n_{spec}+4\))行是注释行, 在程序中无意义, 但必须存在. 
  \item 第(\(n_{spec}+5\))至(\(2n_{spec}+4\))行声明了元素之间的\(sigma\)值对应的矩阵.
  \item 第(\(2n_{spec}+5\))行必须为书写含有``Beta"字符的注释. 若此字符未出现, 则系统将强制把全部\(beta\)值设为1.
  \item 第(\(2n_{spec}+6\))至(\(3n_{spec}+5\))行声明了元素之间的\(beta\)值对应的矩阵.
\end{maineu}}}\\

例如,
\begin{lstlisting}[language={bash},numbers=left]
2 12 6 5
A B
# Epsilon
1.00 1.50
0.50
# Sigma
2.00 1.60
1.76
\end{lstlisting}

下面通过一个例子来演示AIRSS联合 \verb|pp3| 的计算过程.
\begin{lstlisting}[language={bash}]
user@machine_name$ ls 
Al.cell Al.pp
user@machine_name$ cat Al.cell
%BLOCK LATTICE_CART
2 0 0
0 2 0
0 0 2 
%ENDBLOCK LATTICE_CART
 
%BLOCK POSITIONS_FRAC
Al 0.0 0.0 0.0 # Al1 % NUM=8
%ENDBLOCK POSITIONS_FRAC

#MINSEP=1.5
user@machine_name$ cat Al.pp
1 12 6 2.5
Al
# Epsilon
1
# Sigma
2
# Beta
1
user@machine_name$ airss.pl -pp3 -max 3 -seed Al
user@machine_name$ ls -1
Al-43867-3302-1.res
Al-43867-3302-2.res
Al-43867-3302-3.res
Al-43867-3302.cell
Al.cell
Al.pp
user@machine_name$
\end{lstlisting}

\subsection{联合CASTEP弛豫}
官方推荐AIRSS结合CASTEP使用, 且构建了 AIRSS 和 CASTEP 间十分完善的接口. 

要使用 CASTEP 联合 AIRSS 计算, 首先需要将成功安装的 CASTEP 可执行文件 \verb|castep.serial| 或 \verb|castep.mpi| 复制到 AIRSS 的 \verb|bin| 目录中, 并重命名为 \verb|castep|.

使用 CASTEP 联合 AIRSS 计算时, 除了*.cell外, 还需要准备 *.param 文件. *.param 是CASTEP的配置文件, 您可以在其中定义CASTEP计算过程中的必要配置参数, 包括, 计算的类型(结构优化, 自洽, 光学性质计算, 能带计算 等), 电荷, 自旋取向, 截断能, 收敛标准等. 该文件通常由若干行组成, 每一行包含一个 \verb|keyword| 及其相应的赋值.

\noindent\fcolorbox{mygray}[HTML]{EEEEFF}{\parbox{\textwidth}{%
\noindent \textbf{*.param 文件主要有以下特点:}
\begin{maineu}
\item 任何两个 keywords 之间没有书写顺序上的限制.
\item 您可以使用 \# 或 ; 或 ! 又或者单词 COMMENT 来添加注释.
\item *.param 中设定的所有的 keywords 和数据均不区分大小写, 同时, 任何标点符号(除了标明注释内容的符号),多余的空格和任何空行都将被自动忽略.
\item 文件的任何一行中最多只能出现一个 keywords 及其对应参数.
\end{maineu}}}\\

*.param 文件每一行的基本格式均为:
\begin{lstlisting}
[keywords] : [value]
\end{lstlisting}
其中的`\verb|:|'是为了书写美观便于区分内容所加, 程序实际执行时会自动忽略, 您也可以完全不加入这一符号转而用空格代替.

CASTEP中\verb|[keywords]|的定义和使用方法, 可参考: \href{http://www.tcm.phy.cam.ac.uk/castep/documentation/WebHelp/content/modules/castep/keywords/k_main_structure.htm}{CASTEP cell keywords and data blocks}.

AIRSS默认联合CASTEP计算, 因此运行下述命令即可启动结构搜索.
\begin{lstlisting}[language={bash},numbers=left]
user@machine_name$ airss.pl -max 3 -seed Al
\end{lstlisting}

\subsection{联合VASP弛豫}
AIRSS自带了 \verb|airss.pl -vasp| 选项用于联合VASP弛豫计算. 但官方自带的接口在并行计算方面还有待完善. 基于AIRSS, 笔者用一套bash命令集重新编写了AIRSS至VASP的接口, 将其命名为airss4vasp(a4v). 后虽考虑过将此命令集使用python重新编写, 但AIRSS本身无法在Windows上运行, 而复写工程又过于庞大且收效甚微, 因此a4v目前仍然主要基于bash实现.

a4v可以看做是AIRSS的改版, 无需安装原生AIRSS便可独立运行. 其主要基于AIRSS原生的 \verb|buildcell| 模块, 同时内嵌了PBS, NSCC, Slurm等作业提交系统指令, 真正做到了一键提交 AIRSS+VASP 任务的功能, 同时对计算的并行也有较好的支持. 其具体用法可参见项目内部的 \verb|README.md| 文件. \footnote{事实上, 该说明手册也在此项目中.(笑} 

在 *.cell 文件设置方面: a4v增加了对应原子位置弛豫固定的\verb|SD-XYZ|, \verb|SD-XY|, \verb|SD-X|等原子标签; 同时删减了原先buildcell中设置共线自旋数值的全局参数, \verb|#SPIN=|.

\newpage
\section{数据后处理}
\subsection{*.res文件结构}
AIRSS的计算结果全部储存在了 *.res 文件中. 这种 *.res 结构文件最早在 \href{https://strucbio.biologie.uni-konstanz.de/ccp4wiki/index.php/SHELXL}{SHELX} 中使用. 由于一些历史原因被 CASTEP 和 AIRSS 复用. 由于 SHELX 本身是对 Windows 友好的程序, 因此其输入输出文件的书写格式也沿袭了部分Windows文档的特点, 如 倾向使用大写字母, \verb|REM| 代表注释行, 文件使用 \verb|END| 结尾等.

\begin{lstlisting}[language={bash}]
user@machine_name$ cat Al-43867-3302-1.res
TITL Al-43867-3302-2 0.0000000004 60.4852769773 -53.2712053113 0 0 8 (P63/mmc) n - 1
REM
REM in /Users/alex/Documents/ProgramCode/MaterialCalculateProgram/AIRSS/airss-0.9/examples/1.1
REM
CELL 1.54180 2.2 5.2 5.2  86.6 90.0 90.0
LATT -1
SFAC Al 
Al     1  0.2544637028970  0.9316224149716  0.6657635302849 1.0
Al     1  0.7544640475988  0.0982890099295  0.3324301203388 1.0
Al     1  0.2544640470150  0.3482890078890  0.5824301202379 1.0
Al     1  0.2544640470479  0.8482890103459  0.0824301202324 1.0
Al     1  0.7544640476367  0.5982890097930  0.8324301190566 1.0
Al     1  0.7544637023180  0.1816224159838  0.9157635306900 1.0
Al     1  0.7544637024482  0.6816224143044  0.4157635299253 1.0
Al     1  0.2544637030384  0.4316224167828  0.1657635292340 1.0
END
user@machine_name$
\end{lstlisting}

\noindent\fcolorbox{mygray}[HTML]{EEEEFF}{\parbox{\textwidth}{%
\noindent \textbf{AIRSS输出的 *.res 文件各行的含义如下:}
\begin{maineu}
\item 第一行TITL中的第一项是软件分配给该结构的名称标签, 第二项是系统外加静水压(GPa), 第三项是单胞体积, 第四项是每个单胞总的焓(能量), 第五项是原子自旋值的平均值, 第六项是原子自旋绝对值的平均值, 第七项是体系的总原子数, 第八项是体系所在空间群名称, 最后一项是固定字符 n - 1.
\item 之后若干以REM开头的行是注释行, 记录了文件生成的基本信息, 删除后不会有任何影响.
\item 紧接着以CELL开头的行记录了基本的晶胞信息. 其中, 第一项是一个无意义的小数, 这个小数在原来的SHELX程序中是用来记录得到相关结构所用衍射谱的波长, 在AIRSS中锁定为了一个无意义的小数. 第二至四项是晶格常数\(a\;b\;c\), 接下来五至七项是晶角\(\alpha\;\beta\;\gamma\).
\item 下面一行是LATT -1. 这一行在SHELX中用于标定晶格的对称性, 在AIRSS中锁定为固定值-1.
\item 接下来以 SFAC 开头的一行记录了构成体系的全部元素名称, 不同的元素用一个空格隔开.
\item 最后若干行标定了单胞中原子的位置. 这些行中的第一列是元素符号, 第二列指明了该元素在SFAC 行中出现的次序, 第三到五列是该行储存了原子的分数坐标, 最后一列是原子的占据数, 一般设为1.
\item 文件最终以END行结尾.
\end{maineu}}}\\

\subsection{数据批量化处理}
有了计算数据*.res文件后, 就可以使用\verb|ca|指令进行数据处理了. \verb|ca| 是对AIRSS中基本分析套件 \verb|cryan| 的封装. 
\begin{lstlisting}[language={bash}]
user@machine_name$ ca
ca [-R(recurcsive)] [command line arguments for cryan]
user@machine_name$
\end{lstlisting}

\verb|cryan|的使用方法如下:\footnote{为了行文简洁, 对参数的说明做了少许简化, 请自行运行上述指令查看更详细的信息.}
\begin{lstlisting}[language={bash}]
user@machine_name$ cryan

Usage: cryan [OPTIONS]
The str. are read from STDIN, for example:

     cat *.res | cryan -s
     gunzip -c lots.res.gz | cryan -f H2O
     find . -name "*.res" | xargs cat | cryan -m

cryan options (note - the order matters):

 -r,  --rank                          Rank all str.
 -s,  --summary                       Summary
 -e,  --enthalpy <length_scale>       Plot enthalpy vs. pressure
 -f,  --formula <formula>             Select str. of a given com.
 -fc,  --formula_convert <formula>    Attempt to convert.
 -t,  --top [num]                     Output top few results
 -u,  --unite <thresh>                Unite similar str.
 -dr, --distance <rmax>               Distance threshold
 -de, --delta_e <energy>              Ignore str. above energy
 -sd, --struc_dos <smear>             Plot a structural DOS
 -p,  --pressure <pressure>           Additional pressure
 -m,  --maxwell                       Extract the stable com.
 -ph, --pressure_hull                 Ext. the stable str. with P
 -<n>                                 Component <n>
 -xg, --xmgrace                       Plot output with xmgrace
 -c,  --compare <thresh> <structure>  Compare structures
      --delete                        Delete unwanted str.
 -g,  --geometry [thresh]             Calculate the atomic geometry
 -n,  --num_units                     Report n separate str.
 -d,  --dimensionality                Report dD str.
 -cl, --cluster                       No periodic boundary
 -bl, --bondlength                    Maximum bond length
 -bs, --bondscale                     Bond length scaling
 -dm, --deltamodularity               Modularity bias parameter
 -wt, --weight                        Weight the adjacancy matrix
 -ns, --notsymm                       Clusters  point group off
 -sc, --struct_comm <thresh>          Determine the community str.
 -cm, --community                     Output the community str.
 -am, --adjacancymatrix               Output the adjacancy matrix
 -x,  --xyz                           Output clusters in XYZ format
 -o,  --off                           Output polyhedra in OFF
 -al, --alpha                         Construct alpha shapes
 -l,  --long                          Long names for str.
 -h,  --help, -?                      Print usage information
user@machine_name$
\end{lstlisting}

使用\verb|ca|就可以对之前的计算结果进行分析.
\begin{lstlisting}[language={bash}]
user@machine_name$ ca -r > analysis.data
user@machine_name$ cat analysis.data
Al-43867-3302-2   0.00  7.561  -6.659   8 Al  P63/mmc  1
Al-43867-3302-1  -0.00  7.561   0.000   8 Al  P63/mmc  1
Al-43867-3302-3   0.00  7.564   0.005   8 Al  Fm-3m    1
user@machine_name$
\end{lstlisting}

\noindent\fcolorbox{mygray}[HTML]{EEEEFF}{\parbox{\textwidth}{%
\noindent \textbf{上述输出结果中,}
\begin{maineu}
  \item 第一列是AIRSS软件分配给该结构的名称标签
  \item 第二列压力值(GPa)
  \item 第三列是每个化学式结构单元(fu)的体积
  \item 第四列第一行是一个化学式结构单元(fu)的焓值, 之后的几行是不同结构下相对于第一行的焓值
  \item 第五列是单胞中化学式结构单元(fu)的总个数(单胞中fug的个数乘以一个fug中fu的个数.)
  \item 第六列是化学式结构单元(fu)的化学式
  \item 第七列是空间群名称
  \item 第八列是所有搜索结果中出现该结构的次数
\end{maineu}}}\\

如果您认为所列结果过多, 可以使用\verb|-u|选项, 但是要注意, \verb|-u| 一定要排在\verb|-r|之前使用.\footnote{事实上, 所有排在 rank (-r) 任务之后的参数都会被自动忽略.} 如果您仔细看过 \verb|cryan| 的 help 信息就不难发现这样一个提醒: ``note - the order matters''.
\begin{lstlisting}[language={bash}]
user@machine_name$ ca -u 0.01 -r > analysis.data
user@machine_name$ cat analysis.data
Al-43867-3302-2   0.00  7.561  -6.659   8 Al  P63/mmc  2
Al-43867-3302-3   0.00  7.564   0.005   8 Al  Fm-3m    1
user@machine_name$
\end{lstlisting}

指令 \verb|ca -u| 后所跟的数字是一个无量纲的比例. 可以将这一参数简单的做如下理解:  他标定了晶格相似度阈值. 该数值越大, 容忍度越高, 最终展示出来的不同结构越少. 更详细的, 这一参数标定的距离, 是原子结构内部最接近的两个原子之间距离的倍数. 例如, 现有两个结构十分相似, 晶格中最短键长为1.5 \r{A}, 则 \verb|ca -u 0.1| 就意味着, 依次比较两结构对应原子的两两距离, 如果未能发现这些距离差存在大于0.15 \r{A} 的情况, 则标定这两个结构一致. \footnote{这一段描述只是一个粗浅直观的解释, 实际使用的算法要更加复杂且稳定. 详细见源码``airss-0.9.1/src/cryan/src/cryan.f90''2133行.} 该值可根据需求调整, 建议在0.1--0.01之间选择.

\newpage
\appendix
\section{AIRSS安装日志}
\label{sec:airss-install} 
\textbf{下面将以 airss-0.9.0 版本为例, 简要记录AIRSS的安装.}

AIRSS只支持在命令行(Command Line)使用, 且仅能安装在*nix系统中. 安装此软件前, 您最好已经了解 \href{https://www.gnu.org/software/make/manual/}{GNU make} 的使用方法. 当然, 如果您实在对此不感兴趣, 这不是必须的. 前提是您能完全按照以下步骤操作.

\subsection{软件主体安装}
具体的安装分为以下几步, 非必须步骤已使用\verb|*|标出:
\begin{description}
\item [*(I) 建立安装包文件管理系统] 在开始一切安装之前, 建议作为非\verb|root|用户但是有\verb|sudo|权限的您: 在自己能进行任意操作的家目录\verb|~/|中建立一个安装包管理文件夹, 如\verb|~/install_package|; 同时在系统目录\verb|/usr/local|中建立一个存放airss和其他程序二进制可执行文件的目录, 如\verb|/usr/local/| \verb|airss-0.9/bin|. 

之所以这样建议, 是为了减少您安装过程中在系统目录下需要进行的操作, 降低由此可能引发的事故的概率, 同时让安装过程更简洁(避免每个命令都要使用前缀\verb|sudo ...|, \verb|sudo sh -c "...>..."|). 

当然, 您也可以完全不将软件安装在系统目录, 一切都凭您的个人喜好.
\begin{lstlisting}[language={bash}]
user@machine_name$ cd /usr/local/
user@machine_name$ sudo mkdir -p airss-0.9/bin
Password:
user@machine_name$ cd airss-0.9
user@machine_name$ ls -F 
bin/
user@machine_name$ cd
user@machine_name$ mkdir -p  install_package/AIRSS
user@machine_name$ cd install_package
user@machine_name$ ls -F
AIRSS/
\end{lstlisting}

\item [(II) AIRSS安装包下载]您可以访问前文所述\href{https://www.mtg.msm.cam.ac.uk/Codes/AIRSS}{官方网站}下载 airss-0.9.0.

您可以选择在浏览器上下载, 也可以使用\verb|wget|指令.

\begin{lstlisting}[language={bash}]
user@machine_name$ wget -P ~/Downloads https://www.mtg.msm.
cam.ac.uk/files/airss-0.9.tgz
\end{lstlisting}

\item [(III) 拷贝并解压安装包] 将您下载的\verb|airss-0.9-2.tag|拷贝到安装包管理文件夹中, 并使用\verb|tar|解压.
\begin{lstlisting}[language={bash}]
user@machine_name$ cd AIRSS
user@machine_name$ cp ~/Downloads/airss-0.9-2.tgz .
user@machine_name$ tar -zxvf airss-0.9-2.tgz
x airss-0.9/.hg_archival.txt
x airss-0.9/.hgignore
x airss-0.9/LICENCE
x airss-0.9/README
x airss-0.9/VERSION 
...
...
\end{lstlisting}

\item[(IV) 使用 GNU make 指令安装 AIRSS] 使用\verb|make|等指令安装编译安装AIRSS.
\begin{lstlisting}[language={bash}]
user@machine_name$ cd airss-0.9
user@machine_name$ make
(cd src/pp3/src; make)
gfortran -O3 -c ../../common/constants.f90
gfortran -O3 -c cell.f90
gfortran -O3 -c pp.f90
gfortran -O3 -c opt.f90
gfortran -O3 -c pp3.f90
...
...
user@machine_name$ make install > make_install.log 2>&1
user@machine_name$
user@machine_name$ cat make_install.log
(cp src/pp3/src/pp3 bin/)
(cp src/cabal/src/cabal bin/)
(cp src/buildcell/src/buildcell bin/)
(cp src/cryan/src/cryan bin/)
user@machine_name$
\end{lstlisting}

十分鼓励您今后使用\verb|make isntall|指令时, 将其输出重定向到一个记录文件中, 这样会给您卸载软件时提供便利.

\item[*(V)安放可执行文件] \verb|~/install_package/AIRSS/airss-0.9/bin|存放了安装完毕的可执行文件, 将其拷贝至系统目录下.
\begin{lstlisting}[language={bash}]
user@machine_name$ sudo cp -r bin/ /usr/local/airss-0.9/bin
Password:
user@machine_name$ ls /usr/local/airss-0.9/bin
airss.pl     cabal          cell2lammps  crud.pl      
despawn      gulp_relax mc  pp3_relax    psi4_relax   
spawn-slow   tidy.pl        buildcell    castep2res   
check_airss  cryan          gap_relax    lammps2cell  
niggli       press          run.pl       stopairss    
ca           castep_relax   comp2minsep  csymm        
gencell      lammps_relax   pp3          prim         
spawn        symm

\end{lstlisting}

\item[(VI) 设置系统环境变量] 完成以上所有设置后, 您实际上就可以通过使用使用命令\verb|/usr/local/airss-0.9/bin/airss.pl -[option] [parameter] ...|来运行AIRSS了. 为了简便, 可以考虑在\verb|~/.bash_profile|文件中加入如下内容
\begin{lstlisting}[language={bash}]
## Setting PATH for AIRSS
export PATH="/usr/local/airss-0.9/bin:${PATH}"
\end{lstlisting}

修改储存并退出后, 请重新登入终端, 或运行\verb|source|指令完成环境变量的更新.
\begin{lstlisting}[language={bash}]
user@machine_name$ source ~/.bash_profile 
\end{lstlisting}

这样您就可以在系统中的任何路径上执行\verb|airss.pl|等AIRSS的指令了.

\item[(VII) 检查安装情况] 设置好环境变量后, 您可以在\verb|~/install_package/AIRSS/|
\\\verb|airss-0.9/|下输入\verb|make check|指令检查AIRSS安装情况. 
\begin{lstlisting}[language={bash}]
user@machine_name$ make check
(sh bin/check_airss)
Essential:

airss.pl +
run.pl +
crud.pl +
castep2res +
buildcell +
cryan +
pp3 +
cabal +
cellsym - Install cellsym: http://www.tcm.phy.cam.ac.uk/sw
/check2xsf/cellsym.html
symmol - Patch and install symmol: http://www.
ccp14.ac.uk/ccp/web-mirrors/symmol/~pila/symmol.zip
bob - Get Bob!

Recommended:

castep - Install castep: http://www.castep.org/
optados - Install optados: http://www.tcm.phy.cam.ac.uk/
~ajm255/optados/index.html
qhull - Install qhull from package manager, or: 
http://www.qhull.org/
qconvex - Install qhull from package manager, or: 
http://www.qhull.org/
xmgrace - Install grace from package manager or: 
http://plasma-gate.weizmann.ac.il/Grace/
Rscript - Install R/Rscript and ggtern from package manager
or: https://cran.r-project.org/

Optional:

gulp - Install gulp: http://projects.ivec.org/gulp/
cif2cell - Install cif2cell from: http://cif2cell.
sourceforge.net/

Very optional:

lammps - Install lammps: http://lammps.sandia.gov/
hull - Install hull: http://www.netlib.org/voronoi/
hull.html
off_util - Install antiprism: http://www.antiprism.com/
files/antiprism-0.24.1.tar.gz

Pseudopotentials:

pspot - set $PSPOT_DIR to location of the CASTEP pspot 
directory

Spawn file:

.spawn - 

--------------------
Tests run in .check:
--------------------

Running example 1.1 (Crystals):

Al-9002-4643-1   -0.00   7.561    -6.659   8 Al     n/a   1
Al-9002-4643-2    0.00   7.564     0.005   8 Al     n/a   1

Running example 1.2 (Clusters):

Al-9274-4255-2    0.00   615.385  -3.014  13 Al     n/a   1
Al-9274-4255-1    0.00   615.385   0.019  13 Al     n/a   1

Skipping example 3.1 (Gulp)
Skipping example 2.1a (Castep)
\end{lstlisting}       
\end{description}
如果您仔细阅读了上述输出文件, 会发现必要的组件中还有\verb|cellsym|和\verb|symmol|没有安装. 这直接导致了晶体和团簇空间群符号输出为\verb|n/a|. 
\subsection{辅助插件安装}
AIRSS支持的全部插件信息可查询\verb|~/install_package/AIRSS/airss-0.9/|
\verb|README|文件.下面只演示最核心的\verb|cellsym|和\verb|symmol|插件的安装过程.
\begin{description}
\item[(I) 下载插件安装包]
\verb|cellsym|的安装包官方网站是:

\url{http://www.tcm.phy.cam.ac.uk/sw/check2xsf/cellsym.html}

需要注意的是, \verb|cellsym|源码是使用C语言编写的, 安装此程序前, 需要下载并安装库文件\verb|spglib.h|.

\verb|spglib.h|的下载地址是:

\url{http://www.tcm.phy.cam.ac.uk/sw/check2xsf/spglib-1.9.4.tar.gz}

\verb|cellsym|的下载地址是:

\url{http://www.tcm.phy.cam.ac.uk/sw/check2xsf/cellsym.tgz}

\verb|symmol|插件安装包的下载地址是:

\url{http://www.ccp14.ac.uk/ccp/web-mirrors/symmol/~pila/symmol.zip}

您可以通过浏览器下载上述文件, 也可以使用\verb|wget|指令下载.
\begin{lstlisting}[language={bash}]
user@machine_name$ wget -P ~/Downloads 
www.tcm.phy.cam.ac.uk/sw/check2xsf/spglib-1.9.4.tar.gz 
www.tcm.phy.cam.ac.uk/sw/check2xsf/cellsym.tgz 
www.ccp14.ac.uk/ccp/web-mirrors/symmol/~pila/symmol.zip
\end{lstlisting}

\item[(II) 拷贝并解压插件]将您下载的三个压缩包拷贝到安装包管理文件夹中, 并使用\verb|tar|和\verb|unzip|解压.
\begin{lstlisting}[language={bash}]
user@machine_name$ cd ~/Downloads
user@machine_name$ cp cellsym.tar spglib-1.9.4.tar 
symmol.zip ~/install_package/AIRSS
user@machine_name$ cd ~/install_package/AIRSS
user@machine_name$ tar -xvf cellsym.tar
...
user@machine_name$ tar -xvf spglib-1.9.4.tar
...
user@machine_name$ unzip symmol.zip -d symmol
...
user@machine_name$ ls -F
airss-0.9/     airss-0.9-2.tgz   cellsym-0.16a/  
cellsym.tar    spglib-1.9.4/     spglib-1.9.4.tar  
symmol/        symmol.zip
\end{lstlisting}

\item[(III) 编译插件]将解压好的插件按如下顺序操作.

首先安装库文件 \verb|spglib|.使用 GNU make 指令. 
\begin{lstlisting}[language={bash}]
user@machine_name$ cd spglib1.9.4/
user@machine_name$ ./configure
...
user@machine_name$ make
...
user@machine_name$ sudo sh -c 'make install > make_install.log 2>&1'
Password:
user@machine_name$ cat make_install.log 
Making install in src
.././install-sh -c -d '/usr/local/lib'
/bin/sh ../libtool   --mode=install /usr/bin/install -c   libsymspg.la '/usr/local/lib'
libtool: install: /usr/bin/install -c .libs/libsymspg.0.dylib /usr/local/lib/libsymspg.0.dylib
libtool: install: (cd /usr/local/lib && { ln -s -f libsymspg.0.dylib libsymspg.dylib || { rm -f libsymspg.dylib && ln -s libsymspg.0.dylib libsymspg.dylib; }; })
libtool: install: /usr/bin/install -c .libs/libsymspg.lai /usr/local/lib/libsymspg.la
libtool: install: /usr/bin/install -c .libs/libsymspg.a /usr/local/lib/libsymspg.a
libtool: install: chmod 644 /usr/local/lib/libsymspg.a
libtool: install: ranlib /usr/local/lib/libsymspg.a
/Applications/Xcode.app/Contents/Developer/Toolchains/XcodeDefault.xctoolchain/usr/bin/ranlib: file: /usr/local/lib/libsymspg.a(debug.o) has no symbols
.././install-sh -c -d '/usr/local/include/spglib'
/usr/bin/install -c -m 644 arithmetic.h cell.h debug.h delaunay.h hall_symbol.h kgrid.h kpoint.h mathfunc.h niggli.h pointgroup.h primitive.h refinement.h site_symmetry.h sitesym_database.h spacegroup.h spg_database.h spglib.h spin.h symmetry.h version.h '/usr/local/include/spglib'
make[2]: Nothing to be done for `install-exec-am'.
make[2]: Nothing to be done for `install-data-am'.
user@machine_name$
user@machine_name$
user@machine_name$ make install check
...
...
...
PASS: spglib_test
=====================================
Testsuite summary for spglib 1.9.4
=====================================
# TOTAL: 1
# PASS:  1
# SKIP:  0
# XFAIL: 0
# FAIL:  0
# XPASS: 0
# ERROR: 0
=====================================
make[1]: Nothing to be done for `check-am'.
user@machine_name$
\end{lstlisting}

使用\verb|make install check|检查PASS后, 就可以开始编译\verb|cellsym|了.
\begin{lstlisting}[language={bash}]
user@machine_name$ cd ../cellsym-0.16a/
user@machine_name$ make
...
user@machine_name$ ls -all cellsym
-rwxr-xr-x  1 user  groups  53628 Jan 25 12:28 cellsym
user@machine_name$ 
\end{lstlisting}

顺利编译完成后, 会生成一个名为\verb|cellsym|的可执行文件.
注意, \verb|make|执行过程中可能会出现编译警告, 但这并不影响程序执行, 可忽略.

编译并确认生成了\verb|cellsym|文件后, 就可以开始编译另一个插件\verb|symmol|了.
\verb|symmol|是使用Fortran写成的. 在网站上下载的是其源码, 需要编译使其变为可执行文件.
需要注意的是, 原版的\verb|symmol.f|并不兼容AIRSS, 需要为其打上\verb|~/install_package/AIRSS/airss-0.9/misc|中提供的\verb|symmol.patch|补丁.
\begin{lstlisting}[language={bash}]
user@machine_name$ cd ../airss-0.9/misc/
user@machine_name$ cp ../../symmol/symmol.f .
user@machine_name$ ls
symmol.f     symmol.patch
user@machine_name$ patch -p0 symmol.f symmol.patch 
patching file symmol.f
user@machine_name$ gfortran symmol.f -o symmol 
user@machine_name$ ls 
symmol       symmol.f     symmol.patch
user@machine_name$ echo '-o 后跟的文件名一定要是 symmol'
-o 后跟的文件名一定要是 symmol
user@machine_name$ ls -all symmol
-rwxr-xr-x  1 user  group  106800 Jan 25 12:41 symmol
user@machine_name$
\end{lstlisting}

至此, 我们完成了所有插件的编译. 生成了\verb|symmol|和\verb|cellsym|两个可执行文件.

\item[(IV) 将插件导入AIRSS]
这一步的操作十分简单, 将编译好的两个插件复制到系统目录下的\verb|bin/|文件夹即可. 为了以防万一, 可以在安装包管理文件夹保存一个\verb|bin/|的备份
\begin{lstlisting}[language={bash}]
user@machine_name$ pwd
/home/user_name/install_package/AIRSS/airss-0.9/misc
user@machine_name$ cp symmol ../bin/
user@machine_name$ sudo cp symmol /usr/local/airss-0.9/bin
Password:
user@machine_name$ cd ../../cellsym-0.16a/
user@machine_name$ cp cellsym ../airss-0.9/bin/
user@machine_name$ sudo cp cellsym /usr/local/airss-0.9/bin
\end{lstlisting}

\item[(V) 安装最终检查]
回到\verb|airss-0.9|中执行\verb|make|的文件夹. 重新输入\verb|make check|检查安装情况.
\begin{lstlisting}[language={bash}]
user@machine_name$ cd ../airss-0.9
user@machine_name$ make check
(sh bin/check_airss)
Essential:

airss.pl +
run.pl +
crud.pl +
castep2res +
buildcell +
cryan +
pp3 +
cabal +
cellsym +
symmol +
bob - Get Bob!

Recommended:

castep - Install castep: http://www.castep.org/

...
...
...

--------------------
Tests run in .check:
--------------------

Running example 1.1 (Crystals):

Al-14776-403-2  -0.00   7.784  -6.398   8 Al    C2/m    1
Al-14776-403-1   0.00   7.820   0.066   8 Al    P21/m   1

Running example 1.2 (Clusters):

Al-15054-7410-1  0.00   615.385  -3.190  13 Al    Cs    1
Al-15054-7410-2  0.00   615.385   0.006  13 Al    Cs    1

Skipping example 3.1 (Gulp)
Skipping example 2.1a (Castep)
user@machine_name$
\end{lstlisting}
\end{description}

成功输出了晶体的空间群名称! 

至此, 我们完成了AIRSS的基本安装, 您现在已经可以使用AIRSS的pp3模块(默认是CASTEP)进行结构搜索了.

AIRSS是受GPL许可证保护的开源软件. 对此程序您有以下三种权利:
\begin{enumerate}
\item [*]以任何目的运行此程序
\item [*]再复制
\item [*]改进此程序, 并公开发布改进
\end{enumerate}

\subsection{卸载软件}
AIRSS卸载可分为三步:
\begin{description}
\item [(I)卸载spglib] 进入安装包管理文件夹, 使用\verb|make uninstall|卸载spglib.
\begin{lstlisting}[language={bash}]
user@machine_name$ cd ~/install_package/AIRSS/spglib-1.9.4
user@machine_name$ sudo make uninstall
...
user@machine_name$
\end{lstlisting}

\item [(II)删除相关文件夹] 删除系统目录中的bin文件. 您可以选择保留安装文件. 保留安装文件可以在您试图恢复使用AIRSS时提供便利.\footnote{强烈建议您对文件进行删除时, 在离此文件较近的路径上操作, 并杜绝使用绝对路径, 以免打出文章开头提到的的毁灭性指令.}
\begin{lstlisting}[language={bash}]
user@machine_name$ cd /usr/local/
user@machine_name$ sudo rm -ri airss-0.9 
Password:
user@machine_name$ cd ~/install_package/
user@machine_name$ rm -r AIRSS
\end{lstlisting}

\item [(III)恢复PATH变量] 进入\verb|~/.bashrc|文件, 删除修改环境变量的语句即可.
\begin{lstlisting}[language={bash}]
###Setting PATH for AIRSS
export PATH="/usr/local/airss-0.9/bin:${PATH}"
\end{lstlisting}
\end{description}

\end{document}